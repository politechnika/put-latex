\documentclass[polish]{article}

\usepackage[utf8]{inputenc}
\usepackage[OT1]{fontenc}

\usepackage[polish]{babel}

\usepackage[widespace]{fourier}
\usepackage[scaled]{luximono}

\usepackage{fancyvrb}
\usepackage{graphicx}
\usepackage{xcolor}
\usepackage[bookmarks=true,unicode]{hyperref}
\usepackage{url}
\usepackage{varioref}
\usepackage{listings}

% Custom floats
\usepackage{tocloft}
\usepackage{float}

% Set up hyperref.
\hypersetup{
  bookmarksopen=false,
  bookmarksopenlevel=3,
  bookmarksnumbered=true,
%
  colorlinks=false,
  citebordercolor=0.64 0.68 0.86,
  filebordercolor=0.64 0.68 0.86,
  linkbordercolor=0.64 0.68 0.86,
  menubordercolor=0.64 0.68 0.86,
  pagebordercolor=0.64 0.68 0.86,
  urlbordercolor=0.64 0.68 0.86,
  runbordercolor=0.64 0.68 0.86,
%
  pdfpagemode=UseOutlines,
  pdfstartview=Fit,
%
  pdftitle={},
  pdfauthor={},
  pdfkeywords={}
}

% Define colors for listings.
\definecolor{codestrings}{rgb}{0.164,0,1}
\definecolor{codecomment}{rgb}{0.25,0.49,0.37}
\definecolor{codekeywords}{rgb}{0.49,0,0.33}
\definecolor{codebackground}{rgb}{0.95,0.95,0.95}

% Set up defaults
\lstset{
	inputencoding=utf8,
	language=Java,
	extendedchars=true,
	basicstyle=\ttfamily\scriptsize,
	numbers=left,
  numbersep=3pt,
	framexleftmargin=2pt,
  framerule=0pt,
  frame=lines,
	numberstyle=\tiny,
	tabsize=2,
	showstringspaces=false,
	showspaces=false,
  keywordstyle=\bfseries\color{codekeywords},
  identifierstyle=\color{black},
  stringstyle=\color{codestrings},
  commentstyle=\color{codecomment},
  columns=fullflexible,
  abovecaptionskip=\medskipamount,
  belowcaptionskip=\medskipamount,
  backgroundcolor=\color{codebackground},  
}

\RecustomVerbatimEnvironment{Verbatim}{Verbatim}%
    {fontsize=\footnotesize,frame=lines,numbers=left}

\lstnewenvironment{javablock}{\lstset{language=Java}}{}
\lstnewenvironment{xmlblock}{\lstset{language=XML}}{}

\newfloat{Program}{tbp}{lst}

\begin{document}

\noindent
An example of the \texttt{listings} package. Note how certain
keywords are emphasized.

\lstinputlisting[emph={Arrays,asList},emphstyle={\color{red}}]{Example.java}

\bigskip\noindent
And now a block of Java code:

\begin{javablock}
import java.util.*;

/**
 * An example class.
 *
 * @author Dawid Weiss
 */
public final class Example {
	/**
	 * Command line entry point.
	 */
	public static void main(String [] args) {
		if (args.length == 0) {
			System.out.println("Hello world.");
		} else {
			System.out.println(Arrays.asList(args[0]));
		}
	}
}
\end{javablock}

\bigskip\noindent
Finally, a block of XML code:
\begin{xmlblock}
<?processing-instruction content of a processing instruction ?>

<root>
  <!-- A comment -->
  <element attr="value">
  </element>
</root>
\end{xmlblock}

\section{Implementacje algorytmów}

\noindent\ldots algorytm ten implementuje procedura w Javie zawarta w programie~\vref{listing:code1}. 

\begin{Program}[p]
\begin{javablock}
// Search for pattern phrases
BooleanQuery query = new BooleanQuery();
for (int j = 0; j < Math.min(100, featureVector.size()); j++) {
    final TermQuery tk = new TermQuery(
        new Term("keywords", featureVector.get(j).feature));
    tk.setBoost((float) fv.get(j).weight);
    query.add(tk, BooleanClause.Occur.SHOULD);
}
\end{javablock}
\caption{Fragment kodu odpowiadający za obliczenia XXX.}\label{listing:code1}%
\end{Program}
 

\end{document}
