
\documentclass[a4paper,twoside]{article}

\usepackage[utf8]{inputenc}
\usepackage[OT1]{fontenc}
\usepackage[absolute]{textpos}
\usepackage{graphicx}
\usepackage[bookmarks=false,pdffitwindow]{hyperref}

\usepackage{pdfpages}

\title{How to stretch a picture to fill the whole page?}
\author{Dawid Weiss}
\date{April 12, 2006}

\begin{document}

\maketitle

\section{Solution using \textsf{textpos} Package}

Perhaps the simplest way to do it is with \textsf{textpos} package
using the \texttt{absolute} option.

Note that if you are already using \textsf{textpos}, then it may
be difficult to position the box absolutely.

\clearpage\thispagestyle{empty}~%
\textblockorigin{0cm}{0cm}
\begin{textblock*}{\paperwidth}(0cm,0cm)
\noindent \includegraphics[width=\paperwidth]{frontpage.jpg}
\end{textblock*}

\clearpage%
\section{Solution by PDF inclusion}

This solution is perhaps even better (if you're using \texttt{pdflatex}). Just
include a PDF page directly in the generated document. You'll need \textsf{pdfpages}
package for that. To create a PDF from an image (JPG or PNG), use GIMP to save it as
an EPS file and then convert the EPS to a PDF with GhostScript or anything else.

\includepdfmerge{frontpage-eps.pdf,1}




\end{document}

