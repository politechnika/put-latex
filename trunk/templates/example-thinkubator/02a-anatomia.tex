
\chapter{Cel i~zakres}
\label{sec:CeliZakres}
\section{Obecny stan wiedzy dziedzinowej}
W tym rozdziale umieszczono opis podstawowych reguł, do których stosują się
producenci inkubatorów. Zamieszczono też zalecenia, co robić, aby mając już jajo
zmaksymalizować szanse na udane wyklucie. 

\subsection{Jak definiuje się optymalne warunki inkubacji}
Optymalne warunki inkubacji mogą być definiowane jako takie, które prowadzą do
maksymalnej klujności zdrowych zalążków. Ogólnie przyjmuje się, że ewolucja
doprowadziła do takich warunków w~naturalnej inkubacji poprzez dobór naturalny:
warunki środowiska, w~którym żyje dany gatunek, sposób gniazdowania rodzica oraz
jego zachowania inkubacyjne, a~także poprzez adaptacje morfologiczne,
anatomiczne, fizjologiczne i~cząsteczkowe danego gatunku. Wynikłe w~ten sposób
warunki inkubacji dla poszczególnych gatunków ptaków są uważane za
reprezentacyjne czynniki wyznaczające optymalną inkubację. Wśród tych czynników
rozpoznajemy takie, których optymalne wartości zdają się być ogólne dla
wszystkich gatunków. Dla pozostałych próbuje się określić zależności od masy
jaja i~tempa rozwoju embrionalnego.

Poprzez skorelowanie użytych parametrów z~wynikami inkubacji próbuje się
stworzyć uniwersalny model inkubacji. Można go wtedy zastosować do gatunków, co
do których nie ma znajomości potrzebnych ustawień inkubatora (np.: bardzo
rzadkich lub utrudniających badania), a~następnie dalej optymalizować.

\subsection{Temperatura embrionu a~temperatura inkubatora i~jaja w~gnieździe}
Podstawowym problemem w~określeniu warunków sztucznej inkubacji jest różnica
w~sposobie dopływu i~regulacji ciepła. Z~pewnymi wyjątkami w~naturze ptak
przekazuje ciepło przez skórę brzucha (często znajduje się tam specjalne, mocno
ukrwione i~łysiejące na czas składania jaj miejsce) bezpośrednio na wierzch
jaja. Ciepło przepływa do wnętrza jaja, i~równocześnie rozchodzi się po całym
gnieździe. W~zależności od troskliwości ptaka, częstości obracania jaja
i~temperatury otoczenia zmienia się gradient temperatury wewnątrz jaja i~jej
średnia wartość. Z~tego powodu trudno określić na podstawie obserwacji natury
jej wartość (i to dość stałą) optymalną do sztucznej inkubacji.

Różnice w~tych temperaturach są bardzo istotne dla powodzenia procesu --
wpływają na najważniejszą w~procesie inkubacji -- temperaturę embrionu $T_{emb}$.
Ma ona duży wpływ na to, w~jakim tempie (i czy w~ogóle) zarodek będzie się
rozwijał. W~normalnym przedziale temperatury wewnątrz inkubatora wzrost młodego
embrionu jest dużo bardziej podatny na zmiany temperatury (zwiększenie jej
o~1\st{}  powoduje 10\% przyspieszenie rozwoju) niż późnego
embrionu \cite{AA:Princ95}. Korzystając z~tej wiedzy można przyspieszyć inkubację
bez przegrzewania zarodka. 

Na $T_{emb}$ mają wpływ również inne czynniki: 
\begin{itemize}
	\item ochłodzenie wynikające z~parowania,
	\item ciepło wydzielane z~embrionu z~powodu jego metabolizmu,
	\item pojemność cieplna składników jaja,
	\item przewodność cieplna jaja.
\end{itemize}

Temperaturę zarodka $T_{emb}$ mierzy się na kilka sposobów, z~których każdy ma
nierozerwalnie związane ze sobą wady. 
\begin{itemize}
	\item Wiele pomiarów zostało dokonanych poprzez szybkie umieszczenie w~jaju
		termometru. Jajo zostawało wtedy zniszczone i~nie można było prześledzić
		przebiegu $T_{emb}$ podczas całej inkubacji.
	\item Pomiary pomiędzy jajami w~gnieździe nie udzielają jednoznacznej
		odpowiedzi na temat temperatury wewnątrz jaja.
	\item Sztuczne jaja z~umieszczoną wewnątrz sondą nie metabolizują, nie parują
		i~mogą mieć inne parametry cieplne (przewodność, pojemność).
\end{itemize}

Pomiary dla różnych gatunków wskazywały zwykle wartość temperatury w~gnieździe
z~przedziału 33,5\st{} -~35,5\st{}. Biorąc pod uwagę wymienione wcześniej czynniki
zauważa się, że $T_{emb}$ przyjmuje wartość pomiędzy 35\st{} i~37\st{}
i~przyjmuje się przybliżone 36\st{}.

Na początku inkubacji temperatura jaja jest niższa od temperatury otoczenia
o~ok. 0-0,5\st{} z~powodu parowania. W~połowie inkubacji straty ciepła z~powodu
parowania równoważone są poprzez ciepło z~metabolizmu, natomiast pod koniec
inkubacji $T_{emb}$ potrafi być wyższa od temperatury wewnątrz inkubatora nawet
o~2\st{}.

\subsection{Wilgotność inkubatora i~utrata wody}
W dalszym ciągu najczęściej stosowanym urządzeniem do pomiaru wilgotności jest
psychrometr. Pomiar polega na ustawieniu obok siebie termometrów suchego
i~mokrego (którego bańka jest owinięta wilgotną gazą i~nawiewana powietrzem
z~prędkością~$3\frac{m}{s}$). Na podstawie różnicy pomiarów wyznacza się
z~tablic wilgotność względną~$RH$. Jednakże sama wilgotność $RH$ nie ma
znaczenia, ważna jest ilość wody utracona przez jaja podczas inkubacji --
$M_{H_{2}O}$.

Utrata wody jest ważna z~kilku powodów. Po pierwsze, pod koniec inkubacji woda
wydzielana z~powodu metabolizmu łącznie z~naturalną jej utratą przywraca
właściwą ilość wody i~jonów w~tkankach pisklaka. Po drugie, utrata wody skutkuje
w powiększaniu się komory powietrznej, potrzebnej do właściwego rozpoczęcia
oddychania powietrzem atmosferycznym. 

Wykazano, że przeciętna utrata wody podczas inkubacji wynosi ok. 15\%
początkowej masy jaja $M_{egg}$. Zatem, dla inkubacji trwającej $I$ dni, dzienna
utrata wody powinna wynosić

$$m_{H_{2}O} =~\frac{M_{egg} \cdot 0,15}{I}.$$

Należy tak nastawić inkubator, aby utrata wody wynosiła dziennie $m_{H_{2}O}$,
przy uwzględnieniu warunku, że do momentu przeniesienia jaj do klujników powinny
one utracić 13-13,5\% przewidzianej utraty wody.

Dokładniejsze wyliczenia można znaleźć w~literaturze. Korzystają one
z~wyliczonych wartości przepuszczalności pary i~ciśnienia pary nasyconej
wewnątrz jaja. Uwzględniając wszystkie ubytki masy składników jaja (woda,
skorupka, membrana), waga nowo wyklutego pisklaka powinna wynosić 67\% wagi
świeżego jaja.

\subsection{Ustawienie jaja, jego obracanie i~zraszanie}
Wpływ pozycji jaja w~inkubatorze nie jest dobrze poznany. Ogólną regułą jest
układanie jaj ptactwa udomowionego pionowo, spiczastym końcem do dołu, natomiast
ptactwa wodnego poziomo. Dla wielu gatunków nie stwierdzono statystycznie
istotnej różnicy w~klujności dla obu wspomnianych wyżej skrajnych pozycji.
Stwierdzona jest jednak szkodliwość ułożenia jaja ptaka udomowionego spiczastym
końcem do góry -- komora powietrzna ulega wtedy szkodliwej deformacji. 

Ilość obrotów jaja nie jest dokładnie określona. Literatura wymienia częstość
obracania na poziomie od jednego na godzinę do jednego na dzień. Badania nad
konkretnymi gatunkami wskazują na znaczne różnice w~optymalnej częstości
obrotów. Wartością, którą można bezpiecznie przyjąć, jest 6~do 12 obrotów
dziennie.

Optymalne kąty obrotów są bardzo słabo zbadane. Producenci inkubatorów stosują
wartości pomiędzy $\pm 30$ a~$\pm 45$ stopni. Bardzo dobre wyniki otrzymuje się
w inkubatorach, w~których jaja są obracane na rolkach wzdłuż ich najdłuższej
osi; taki sposób obracania zdaje się być najbardziej zbliżony do naturalnego.

Częstą praktyką dla jaj ptactwa wodnego jest codzienne zraszanie jaj wodą.
Udowodniono, że zwiększa to klujność wielu gatunków. Ostatnie badania wykazują,
że spryskiwanie można czasem zastąpić okresowym przechładzaniem jaj. Jak do tej
pory działanie zraszania jaj nie zostało wyjaśnione. Przypuszczenia kierowane są
na proces absorpcji wapnia ze skorupki.

\subsection{Wymiana gazowa i~wentylacja inkubatora}
Atmosfera wokół jaj zależy od kilku czynników. Jednymi z~nich są wspomniane już
wilgotność względna $RH$ oraz ciśnienie pary wodnej w~inkubatorze. Parametry te
wpływają na rozcieńczenie innych gazów. Przykładowo, na wysokości równej
poziomowi morza, gdzie ciśnienie barometryczne $P_b$ wynosi 760 Tor,
w~inkubatorze wilgotność bezwzględna $P_{H_{2}O}$ dla temperatury 37,5\st{} i~$RH
= 60\%$  wynosi 28,7 Tor, czyli 3,8\% całego $P_b$. Zatem ciśnienie tlenu
$P_{O_2}$ spada ze 159 Tor (normalnie 20,9\% powietrza) do 153 Tor ($20,1\%$),
zatem jest od początku niższe o~4\%.

Kolejnym czynnikiem jest wentylacja inkubatora  -- $V_i$, czyli prędkość
wprowadzania świeżego powietrza do wewnątrz inkubatora. Nie chodzi tu jednak
o~prędkość cyrkulacji powietrza wewnątrz, co jest nawiasem mówiąc również bardzo
ważne, bo zapobiega tworzeniu się ,,kieszeni'' nierównego rozkładu gazów
i~temperatury. Wentylacja spełnia 4~role:
\begin{itemize}
	\item utrzymuje $P_{O_2}$ wysokie pomimo zużycia tlenu przez zarodki,
	\item utrzymuje $P_{CO_2}$ niskie pomimo wydzielania go przez zarodki,
	\item zapobiega gromadzeniu się odparowanej z~jaj wody i~podnoszeniu się
		$P_{H_{2}O}$,
	\item zapobiega przegrzaniu się zarodków z~powodu wydzielania przez nie ciepła
		z~metabolizmu.
\end{itemize}

Każda z~tych ról może wymagać innego $V_i$. Dokładne wyliczenia można znaleźć
w~literaturze.

\subsection{Nastawy klujnika}
Proces wykluwania jest skomplikowany i~niebezpieczny. Potrzeba dużego nakładu
energii i~koordynacji wielu czynności, aby mógł zakończyć się powodzeniem.
Czynności te wymagają od otoczenia dodatkowego $O_2$ i~generują dodatkowe
ciepło.  Z~drugiej strony, kiedy skorupka jest już pęknięta, uwalniane są
względnie duże ilości płynu, które chłodzą pisklaka. Nie ma prostego modelu tego
procesu, który mógłby posłużyć do zminimalizowania obciążeń młodego ptaka.
Czasem, przy jajach z~małą przewodnością, zaleca się wiercić otwory w~komorze
powietrznej na krótko przed przyjściem pisklaka na świat. Podnosi to tempo
wymiany gazowej oraz wymiany pary wodnej i~ciepła, co zmniejsza gwałtowność
mającego nastąpić wstrząsu. 

Pozostałe procesy zachodzące w~fazie wykluwania to:
\begin{itemize}
	\item wchłonięcie reszty żółtka i~wcielenie pęcherzyka żółtkowego,
	\item stopniowe przechodzenie z~wymiany gazowej dyfuzyjnej (przez membranę
		i~skorupkę) na aktywne konwekcyjne oddychanie przez płuca,
	\item stopniowe przechodzenie z~jednego, równoległego obiegu krwi (krew żylna
		i~tętnicza mieszana w~sercu) na szeregowy,
	\item stopniowe zmiany w~powinowactwie do tlenu i~równowadze kwasowo-zasadowej
		krwi,
	\item stopniowa zmiana pojemności termoregulacyjnej,
	\item inne.
\end{itemize}

Na końcu inkubacji wychodzą na jaw łączne efekty błędów podczas procesu, takie
jak nieodpowiednio zbilansowany udział wody i~energii. Dlatego ok. 50\%
przypadków śmiertelności zarodków zdarza się właśnie podczas klucia. Bardzo
trudno znaleźć przepis na optymalne ustawienia klujnika.  Literatura jest w~tym
względzie sprzeczna. Najprościej i~najbezpieczniej jest nie zmieniać warunków na
ten czas (naśladując w~ten sposób naturę), wyłączając jedynie obracanie jaj
i~podnosząc wentylację. 

%[[Możemy dodać sporo linków do bibliografii w~tym dokumencie]]

