
\cvsid{$Id$}

\chapter{Narzędzia}

Pracując pod systemem Windows, polecam:
\begin{itemize}
    \item MikTeX, \url{http://www.miktex.org/},
    \item JEdit, \url{http://www.jedit.org/},
    \item Ghostview, Ghostscript (podgląd dokumentów PDF bez blokowania pliku):
        \url{http://www.cs.wisc.edu/~ghost/}. 
\end{itemize}

Po zainstalowaniu tych narzędzi wystarczy wykonać polecenie \texttt{compile.bat} (który
jest skryptem wsadowym dla Windows). Dla tych, którzy wolą nieco automatyzacji --- skrypt
\texttt{latexmk}, który jest w MikTeXu (a który potrzebuje zainstalowanego Perla) jest
również bardzo wygodny: \texttt{latexmk -pdf -pvc main.tex}.



\chapter{Tekst}
\chaptermark{Tytuł rozdziału, jeśli pełen się nie mieści\ldots{}}{}

\section{Struktura dokumentu}

Praca składa się z rozdziałów (\texttt{chapter}) i podrozdziałów (\texttt{section}).
Ewentualnie można również rozdziały zagnieżdzać (\texttt{subsection}, \texttt{subsubsection}),
jednak nie powinno się wykraczać poza drugi poziom hierarchii.



\section{Edycja tekstu pracy}

\subsection{Akapity i znaki specjalne}

Każdy akapit to po prostu blok tekstu. Nieważne jak sformatowany -- to zrobi już
system $\LaTeX$.

Akapity rozdziela się od siebie przynajmniej jedną pustą linią. Podstawowe
instrukcje, które się przydają to \emph{wyróżnienie pewnych słów}. Można również
stosować \textbf{styl pogrubiony}, choć nie jest to zalecane.

Należy pamiętać o zasadach polskiej interpunkcji i ortografii. Po spójnikach 
jednoliterowych warto wstawić znak tyldy ($\sim$), który jest tak zwaną
,,twardą spacją''. i powoduje, że wyrazy nią połączone nie będą rozdzielane
na dwie linie tekstu.

Polskie znaki interpunkcyjne różnią się nieco od angielskich: ,,polski'', a to jest
``angielski''. W źródle tego tekstu będzie widać różnicę.

Proszę również zwrócić uwagę na znak myślnika, który może być pauzą ,,---'' lub
półpauzą: ,,--''. Należy stosować je konsekwentnie. Do łączenia wyrazów używamy
zwykłego ,,-'' (\emph{północno-wschodni}), do myślników --- pauzy lub półpauzy.
Inne zasady interpunkcji i typografii można znaleźć w słownikach.



\subsection{Wypunktowania}

Wypunktowanie z cyframi:
\begin{enumerate}
    \item to jest punkt,
    \item i to jest punkt,
    \item a to jest ostatni punkt.
\end{enumerate}

\noindent
Po wypunktowaniach czasem nie warto wstawiać wcięcia akapitowego. Wtedy przydatne jest
polecenie \texttt{noindent}. Wypunktowanie z kropkami (tzw.~\emph{bullet list}) wygląda tak:
\begin{itemize}
    \item to jest punkt,
    \item i to jest punkt,
    \item a to jest ostatni punkt.
\end{itemize}

\noindent
Warto porobić sobie makrodefinicje, które wstawią od razu odpowiednie bloki
\emph{begin-end}. Wypunktowania opisowe właściwie niewiele się różnią:
\begin{description}
    \item[elementA] to jest opis,
    \item[elementB] i to jest opis,
    \item[elementC] a to jest ostatni opis.
\end{description}



\subsection{Inne procedury}

\LaTeX{} jest językiem programowania, więc można właściwie zrobić w nim wszystko.
W internecie jest mnóstwo podręczników, które pokazują jak i co.

