\chapter{Analiza wyników testu}\label{chap:testresultanalysis}
Rozdział ten omawia analizę zebranych wyników z przeprowadzonej procedury testu.
Na uwagę zasługuje fakt, iż analiza może zostać dokonana w dowolnym momencie po zakończeniu testów,
tzn.: nie jest wymagane istnienie połączeń klientów RTE. Wszystkie wyniki bowiem po zakończeniu testu,
są przekazywane na serwer i tam pamiętane.
\section{Analiza}
Jak już wspomniano, ostatnim etapem testowania jest analiza wyników procedury testu. 
Analiza ta jest wykonywana na podstawie:
\begin{itemize}
\item zawartości plików ,,testresult.zip'' uzyskanych z RTE biorących udział w testach,
\item zawartości pliku ,,test.metadata'',
\item analizy modelu obciążenia,
\item wartości parametrów skalujących opisanych w modelu testu,
\item liczby RTE biorących udział w procedurze testowej.
\end{itemize}
Efektem analizy dokonanej przez serwer jest wygenerowanie raportu tekstowego
oraz histogramów czasów odpowiedzi.
\section{Generowanie raportu}\label{sect:textraport}
Raport z testu obejmuje cztery typy testów omówione w punkcie \ref{sect:testtypes}.
Typy te dzielą raport na cztery sekcje:
\begin{enumerate}
\item Wyniki testu obciążeniowego operacji,
\item Wyniki testu obciążeniowego transakcji,
\item Wyniki testu proporcjonalnościowego operacji,
\item Wyniki testu proprocjonalnościowego transakcji.
\end{enumerate}
W sekcji 1 wymienione są wszystkie operacje zdefiniowane w modelu obciążeniowym
w kolejności od operacji najbardziej kosztownej do najmniej. Dla każdej operacji 
podano szereg właściwości:
\begin{itemize}
\item liczbę operacji przypadających na pojedynczy RTE,
\item liczbę wszystkich operacji,
\item średni czas wykonania wszystkich operacji na pojedynczym RTE -- jest to czas
wykonania operacji przydzielonych dla pojedynczego RTE,
\item medianę dla tego czasu,
\item standardowe odchylenie tego czasu,
\item oraz średni czas wykonania pojedynczej operacji.
\end{itemize}
Przykładowy fragment raportu tekstowego z tej sekcji może wyglądać następująco:
\begin{codeblock}
*********************************************************************************
Operation name: SELECT FROM TBL_PRODUCT
Operation type: SELECT
Operation definition:  SELECT * FROM TBL_PRODUCT WHERE PRICE >= ? AND PRICE <= ? AND CATEGORY_ID = ?
Operation description: Wyszukiwanie produktów po cenie i kategorii

Number of test operations per RTE:  5162
Total number of test operations:    41296

Avarage RTE time:               22s 18ms 457us
Median RTE time:                25s 53ms 151us
Standard deviation RTE time:    6s 550ms 933us

Single operation time avarage:  4ms 265us
*********************************************************************************
\end{codeblock}
,,Single operation time avarage'' jest średnim czasem wykonania pojedynczej operacji, czas ten nie jest jednak zmierzony,
lecz wyliczony. Wynika to z faktu, iż dla testów obciążeniowych czasy mierzone są przed rozpoczęciem i po zakończeniu
bloku operacji. Dzięki temu uzyskano większe obciążenie bazy danych. ,,Avarage RTE time'' jest średnim czasem wykonania na RTE
bloku operacji.

%Dla tego typu testów ze względu na liczbę operacji, czas nie jest mierzony dla pojedynczej operacji, a dla grupy operacji
%na pojedynczym RTE, w związku z tym nie możemy wyznaczyć wariancji, mediany czy odchylenia standardowego
%dla średniego czasu wykonania pojedynczej operacji. 
%Czas ten nie jest mierzony ze względu na minimalny 
%czas trwania pojedynczej operacji, w stosunku do którego błąd pomiaru jest znaczący. 
%Podczas pomiaru czasu
%należy bowiem uwzględnić szereg czynników takich jak występowanie wątków i procesów pomiędzy, którymi system operacyjny jest
%przełączany, a na który bezpośrednio nie mamy wpływu w systemie wielozadaniowym. 

%Rozważmy pewien przykład: załóżmy, że testom zostaje poddanych $n = 1000$ operacji, 
%przy pomiarze czasu oddzielnie dla każdej operacji otrzymujemy 1000 wyników, 
%każdy wynik obarczony jest pewnym błędem $\bigtriangleup_i$. Oznaczmy rzeczywisty czas trwania operacji
%przez $t_i$. Jeżeli $t_i \approx \bigtriangleup_i$ to $\sum_{i=1}^n(\bigtriangleup_i + t_i) \neq \sum_{i=1}^n(t_i)$.
%Tylko w sytuacji, gdy $t_i \gg \bigtriangleup_i$ to $\sum_{i=1}^n(\bigtriangleup_i + t_i) \approx \sum_{i=1}^n(t_i)$.
%Widzimy zatem, iż w sytuacji, gdy mierzony czas jest bliski granicy błędu należy stosować pomiar grupowy,
%w takiej sytuacji mamy pojedynczy czas $\sum_{i=1}^n(t_i) + \bigtriangleup$, gdzie $\bigtriangleup$ to
%błąd pomiaru. 

%Innym powodem wyboru tej metody pomiarowej dla testów obciążeniowych była chęć wygenerowania
%możliwie jak największego obciążenia. Pomiar czasów przed i po każdej operacji, na każdym RTE
%powodowałby chwilowe zmniejszenie obciążenia DBMS.

W sekcji 2 wymienione są wszystkie transakcje zdefiniowane w modelu obciążeniowym
w kolejności od transakcji najbardziej kosztownej do najmniej. Dla każdej transakcji 
podano szereg właściwości, które są ekwiwalentne do właściwości operacji omówionych w sekcji 1.
Przykładowy fragment raportu tekstowego może wyglądać następująco:

\begin{codeblock}
*********************************************************************************
Transaction name:        Login/Logout
Transaction description: Sprawdzenie uprawnień

Number of test transactions per RTE:  200
Total number of test transactions:    1600

Avarage RTE time:                 2s 375ms 948us
Median RTE time:                  2s 445ms 641us
Standard deviation RTE time:      257ms 584us

Single transaction time avarage:  11ms 879us
*********************************************************************************
\end{codeblock}

Jak już wspomniano sekcje 1 i 2 odnoszą się do testów obciążeniowych operacji i transakcji,
dwie kolejne sekcje raportu dotyczą testów proporcjonalnościowych. W przypadku testów proporcjonalnościowych
czas mierzony jest niezależnie dla każdej operacji/transakcji.

Sekcja 3 omawia wyniki testu proporcjonalnościowego operacji. W sekcji tej wymienione zostały
wszystkie operacje w kolejności od najbardziej kosztownej do najmniej. Dla każdej operacji 
podano szereg właściwości:
\begin{itemize}
\item liczbę operacji przypadających na pojedynczy RTE,
\item liczbę wszystkich operacji,
\item średni czas wykonania pojedynczej operacji,
\item medianę dla tego czasu.
\item odchylenie standardowe tego czasu,
\item oraz najlepszy i najgorszy uzyskany czas.
\end{itemize}
Przykładowy fragment z tej sekcji może wyglądać następująco:
\begin{codeblock}
*********************************************************************************
Operation name: DELETE FROM TBL_PRODUCT
Operation type: DELETE
Operation definition:  DELETE FROM TBL_PRODUCT WHERE PRODUCT_ID = ?
Operation description: Usunięcie produktu

Number of test operations per RTE:   2
Total number of test operations:    16

Avarage RTE time:                393ms 264us
Median RTE time:                 363ms 636us
Standard deviation RTE time:     263ms 988us

Best RTE time:                    69ms 416us
Worst RTE time:               1s  15ms 137us
*********************************************************************************
\end{codeblock} 
Średni czas, wariancja, odchylenie standardowe i mediana dotyczą
bezpośrednio czasu wykonania operacji, gdyż dla tego typu testów czasy mierzone
były niezależnie dla każdej operacji/transakcji.

Sekcja 4 omawia wyniki testu proporcjonalnościowego transakcji. W sekcji tej wymienione zostały
wszystkie transakcje w kolejności od najbardziej kosztownej do najmniej. Dla każdej transakcji 
podano szereg wartości analogicznych do tych z sekcji 3. Poniżej zamieszono przykładowy fragment:
\begin{codeblock}
*********************************************************************************
Transaction name:        Create order
Transaction description: Utworzenie zamówienia

Number of test transactions per RTE:   200
Total number of test transactions:     1600

Avarage RTE time:                 328ms 103us
Median RTE time:                  237ms 514us
Standard deviation RTE time:      312ms 069us

Best RTE time:                     39ms 886us
Worst RTE time:                2s 381ms 393us
*********************************************************************************
\end{codeblock}

Powyżej została omówiona struktura raportu tekstowego, raport ten w graficznej wersji serwera benchmarku, 
został rozszerzony o graficzne histogramy czasu trwania operacji i transakcji dla
testu proporcjonalnościowego. Bliższe informacje na ten temat znajdują się w rozdziale \ref{chap:GUI}.

Przykładowy raport znajduje się w załączniku.

\section{Podsumowanie}
Wygenerowany raport umożliwia rozpoznanie transakcji/operacji najbardziej kosztownych dla
użytego modelu bazy danych i obciążenia. Jeżeli zatem modele te obrazują rzeczywisty 
system bazodanowy, który jest analizowany, można użyć raportu w celu optymalizacji
struktury bazy danych, bądź używanych operacji. Jeżeli zaś test miał na celu porównanie
wydajności różnych DBMS, analiza kosztów poszczególnych transakcji/operacji pozwala
wyciągnąć wnioski co do różnic w wydajności. Poza tymi dwoma zastosowaniami bieżący benchmark
może być używany do porównywania wydajności różnych sterowników JDBC dla jednego,
konkretnego DBMS. W tym celu wystarczy wykonać procedurę testu dwukrotnie, za każdym
razem modyfikując w modelu testu nazwę sterownika JDBC, który ma zostać użyty. Porównanie
wyników obu testów dostarcza wiedzy odnośnie różnic wydajnościowych jakie mogą istnieć pomiędzy
różnymi wersjami sterowników JDBC dla jednego DBMS.

Z powyższych uwag wynika, iż niniejszy benchmark może mieć szereg zastosowań, których 
z reguły tradycyjne benchmarki nie posiadają. Jest zatem od nich dużo bardziej uniwersalny.
