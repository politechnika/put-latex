
\documentclass[compress]{beamer}

\usepackage[utf8]{inputenc}
\usepackage[OT4]{fontenc}
\usepackage{dwlecture}
\usepackage{soul}

\title{Obliczenia rozproszone w~laboratoriach Instytutu~Informatyki~(na~piechotkę)}
\author{Dawid Weiss}
\date{2006--2007}

\newcommand{\terminal}[1]{{\color{blue}{#1}}}

\AtBeginSection[] {
  \begin{frame}<beamer>
    \framesubtitle{\insertpart}
    \tableofcontents[sectionstyle=show/shaded,subsectionstyle=show/shaded/hide]
  \end{frame}
}

\AtBeginSubsection[] {
  \begin{frame}<beamer>
    \framesubtitle{\insertpart}
    \tableofcontents[sectionstyle=shaded,subsectionstyle=show/shaded/hide]
  \end{frame}
}

\begin{document}

\begin{frame}[plain]
    \titlepage
\end{frame}

\section{Założenia}

\begin{frame}
    \frametitle{Założenia}

    \begin{itemize}
        \item Programy pod system SuSE Linux (lub \emph{cross-platform}).
        \item Konto do serwera \texttt{sirius}/ \texttt{polluks} (a tym samym do
        maszyn w laboratoriach).
        \item Znajomość poleceń powłoki systemu operacyjnego Linux.
    \end{itemize}

    \medskip    
    \begin{itemize}
        \item Użycie jednego z komputerów jako głównego (\emph{master}).
        \item Użycie serwerów (końcówek) obliczeniowych (\emph{slaves}).
    \end{itemize}
\end{frame}

\begin{frame}
    \frametitle{Środowisko}

    Linux computers at the Institute use SuSE Linux and provide the following facilities:
    \begin{itemize}
        \item shared login system (LDAP client); you have the same user name and password on all
        computers,
        \item shared home folder (mounted network file system); this way you ``see'' the same home folder
        no matter on which machine you log on.
    \end{itemize}

    The above features make it possible to have a uniform, password-less SSH access to all machines. Follow 
    the steps from \texttt{cs-infrastructure} script to setup such an access.
\end{frame}


\section[SSH]{SSH, klucze publiczne, wykonywanie poleceń}

\begin{frame}
    \frametitle{Zdalne wykonywanie poleceń w systemie Linux}

    \begin{block}{}
    Wykonanie polecenia zdalnego polega na uruchomieniu programu
    \texttt{ssh} z autoryzacją przy pomocy \emph{klucza prywatnego}.
    \end{block}
\end{frame}

\begin{frame}
    \frametitle{Generowanie i instalacja klucza (bez hasła)}

    \begin{itemize}
        \item Generacja: \terminal{ssh-keygen -b 1024 -t rsa -f mykey} 
              \\ $\to$ \texttt{mykey}, \texttt{mykey.pub}
        \item Instalacja klucza publicznego (slave): 
              \\ \terminal{cat mykey.pub \textgreater\hspace{-2pt}\textgreater .ssh/authorized\_keys2}

              \medskip
              Zwróć uwagę: współdzielony katalog \texttt{home} sprawia, że \emph{wszystkie}
              komputery w laboratoriach mają od razu zainstalowany ten certyfikat.
              
        \item {\color{red} prosze sprawdzic, czy katalog \texttt{.ssh} ma prawa \texttt{700}
        (tylko dla uzytkownika).}
    \end{itemize}
\end{frame}

\begin{frame}
    \frametitle{Logowanie przy użyciu klucza prywatnego}

    \begin{itemize}
        \item Zalogowanie się na któryś z komputerów (Linux).
        \item Odpalenie powłoki ,,agenta'': \terminal{ssh-agent bash}
        \item Dodanie klucza prywatnego: \terminal{ssh-add mykey}
        \item Sprawdzenie (może poprosić o weryfikację sygnatury hosta): \\
        \terminal{ssh -2 -l \textit{username} polluks.cs.put.poznan.pl "ls -l; pwd"}
        \item W praktyce: \\
        \terminal{ssh -2 -l \textit{username} \textit{slave-host} "\textit{command}"}
    \end{itemize}
\end{frame}

\begin{frame}
    \frametitle{Ułatwienia}

    Cezary Sobaniec oraz Ania i Jacek Kobusińscy udostępnili skrypty, które uruchamiają 
    (sekwencyjnie lub równolegle) \emph{identyczne} polecenie na wielu hostach, 
    zawartych w pliku konfiguracyjnym. 
    
    \medskip
    Plik konfiguracyjny zawiera jeden host w każdej linii.

    \medskip
    Tym samym sekwencyjne sprawdzenie np.~ilości miejsca na wszystkich końcówkach ogranicza się 
    do wykonania polecenia: \\ \terminal{./rshseq.sh computing.hosts df -h /tmp}.
\end{frame}


\section{Obliczenia i agregacja wyników}

\begin{frame}
    \frametitle{Przeprowadzenie obliczeń}

    \begin{enumerate}
        \item Przygotuj program (skrypt) tak, by można było przekazać ,,paczki'' danych (np.~jako
        parametry z linii poleceń).

        \item Zgromadź listę adresów komputerów, które będą wykorzystane (,,slaves''). Komputery
        w laboratoriach mają adresy \texttt{lab-142-X} oraz \texttt{lab-143-X}; nie wszystkie są sprawne.

        \item Utwórz skrypt powłoki na komputerze ,,master'', obliczający fragment zadania
        na każdym z hostów.

        \item Skopiuj wynik działania do katalogu domowego (NFS).
    \end{enumerate}
\end{frame}

\begin{frame}
    \frametitle{Elementy wymagające uwagi}

    \begin{itemize}
        \item Programy lub dane zajmujące dużo miejsca. \\
            $\to$ skopiować na końcówki (kat.~\texttt{/tmp/user}).
        \item Programy lub dane używające dysku. \\
            $\to$ nie używać katalogu domowego.
        \item Komputery popsute/ niesprawne. \\
            $\to$ sprawdzić wyniki/ przejrzeć logi.
        \item Warto uzgodnić wykorzystanie komputerów z administratorami/
        Czarkiem Sobańcem.
    \end{itemize}
\end{frame}

\begin{frame}[plain]
    Podziękowania dla Cezarego Sobańca, Ani i~Jacka Kobusińskich za wskazówki i~pomoc.
\end{frame}

\end{document}
